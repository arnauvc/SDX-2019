\documentclass[a4paper, 10pt]{article}
\usepackage[utf8]{inputenc} % Change according your file encoding
\usepackage{graphicx}
\usepackage{url}

%opening
\title{Seminar Report: [Chatty]}
\author{\textbf{Arnau Valls, Aniol Gamiz, Nil Magnusson}}
\date{\normalsize\today{}}

\begin{document}

\maketitle

\begin{center}
  Upload your report in PDF format.
  
  Use this LaTeX template to format the report.
  
	A compressed file (.tar.gz) containing all your source code files must be submitted together with this report\footnote{Describe in the report any design decision required to understand your code (if any)}.
\end{center}



\section{Introduction}

\begin{align}


\item{Aquest laboratori consisteix en crear un sistema distribuit en el qual diferents usuaris es puguin comunicar a través d'un xat. En aquest cas el sistema distribuit consisteix en tenir el servidor principal i els clients, tots en el mateix xat, en maquines o instancies de Erlang diferents}

\end{align}

\section{Experiments}

\begin{itemize}


\end{itemize}

\newpage

\section{Open questions}

\begin{itemize}



\textbf{1 Server Open Questions:}\\
\\
\textbf{i) Does  this  solution  scale  when  the  number  of  users
increase?\\}No, perquè amb més usuaris, el servidor haura de repartir els missatges a més usuaris i per tant trigara més temps en repartir els missatges de cada usuari.\\
\\
\textbf{ii) What happens if the server fails?\\} Si el server falla tot el xat cau i els clients no es podran enviar missatges.\\
\\
\textbf{iii) Are  the  messages  from  a
single client guaranteed to be delivered to any other client in the order they were
issued? \\} Sí, el middleware garenteix l'ordenació FIFO entre missatges.\\ 
\\
\textbf{iv) Are the messages sent concurrently by several clients guaranteed to be delivered
to any other client in the order they were issued?\\} No, perquè la latencia entre els diferents clients i el servidor pot ser diferents i els missatges poden arribar en ordre diferent.\\
\\
\textbf{v) Is it possible that a client
receives a response to a message from another client before receiving the original
message from a third client?\\} No, perquè tots els missatges passen per un servidor per tant tots els clients veuran els mateixos missatges a la vegada.\\
\\
\textbf{vi) If a user joins or leaves the chat while the server
is broadcasting a message, will he/she receive that message?\\} Si el client abandona el xat mentre el servidor fa el broadcast si que rebrà el missatge ja que encara esta a la llista de clients, en canvi si s'afegeix al xat mentre el servidor fa el broadcast no rebrà el missatge ja que encara no esta a la  llista de clients.\\
\\

\newpage

\textbf{2 Servers Open Questions:}\\}
\\
\textbf{i) What happens if a server fails?\\} Els clients registrats al servidor que caigui deixaran de tenir access al xat.\\
\\
\textbf{ii) Do your answers
to previous questions iii, iv, and v still hold in this implementation?\\} Sí, perquè a efectes pràctics tenir fins a 2 servers representa el mateix que tenir un ja que si tenim 3 nodes, 2 d'ells estaran al mateix servidor per tant rebran els mateixos missatges i en el mateix ordre. \\
\\
\textbf{iii) What
might happen with the list of servers if there are concurrent requests from servers
to join or leave the system?\\} Els servers que hagin fet la petició a la vegada en diferents servidors no es veuran entre ells ja que al ser a la vegada els updates no arriben abans que el server join reques.t\\
\\
\textbf{iv) What are the advantages and disadvantages of
this  implementation  regarding  the  previous  one?\\} Una avantatge obvia és que escala molt millor ja que la feina es pot repartir entre més servidors, també és més tolerant a fallades ja que si cau un servidor només es perdran els clients d'aquell servidor. Com a defecte podriem dir que és més costós d'implementar ja que s'han de controlar més missatges.\\
\\



\end{itemize}

\section{Personal opinion}

\item{Creiem que és una pràctica molt interessant ja que es pot veure clarament el funcionament dels sistemes distribuits, és molt complicada així entres en les bases de'Erlang. Nosaltres la recomanariem basicament perquè creiem que són les bases bàsiques per treballar amb Erlang i els sistemes distribuits.}


\end{document}
