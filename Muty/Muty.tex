\documentclass[a4paper, 10pt]{article}
\usepackage[utf8]{inputenc} % Change according your file encoding
\usepackage{graphicx}
\usepackage{url}

%opening
\title{Seminar Report: [Muty]}
\author{\textbf{Arnau Valls, Aniol Gamiz, Nil Magnusson}}
\date{\normalsize\today{}}

\begin{document}

\maketitle

\begin{center}
  Upload your report in PDF format.
  
  Use this LaTeX template to format the report.
  
	A compressed file (.tar.gz) containing all your source code files must be submitted together with this report\footnote{Describe in the report any design decision required to understand your code (if any)}.
\end{center}



\section{Introduction}

\begin{align}


\item{Aquest laboratori consisteix en crear un sistema distribuit en el qual diferents processos inten accedir a una regió critica i ho fan amb exclusió mutua. Realitzem 3 versions, la primera consisteix en que cada procés intenta accedir i només envia "ok" a un altra proces que vulgui accedir quan ell mateix surti de la zona d'exlcusió, la segona es donen prioritat als processos amb el que un procés que vuglui accedir si li arriba una petició d'un procés més prioritati li haurà de donar el "ok" i per últim una versió en la que es té en compte el temps en el qu el procés ha fet la petició.}

\end{align}

\section{Experiments}
\begin{itemize}
\textbf{i) Provant amb diferents paràmetres de \textit{Sleep} i \textit{Work}}\\
El primer experiment que vam realitzar va ser canviar aquests dos paràmatres abans d'executar el programa. Quan canviem el paràmetre \textit{Sleep} ens quedem adormits més/menys estona abans d'intentar agafar el Lock i quan canviem el paràmetre \textit{Work} ens estem més estona en la zona crítica "treballant" un cop l'agafem.
\\
\\
\textbf{ii) Cada parella \textit{Worker-Lock} en una instància Erlang diferent}\\
El segon experiment realitzat ha sigut el de fer que cada parella \textit{Worker-Lock} s'executi en una instància pròpia d'Erlang. Aquests canvis els hem fet en el codi del \textit{muty.erl}. Concretament, cada vegada que necessitem fer un \textit{register}, haurem de fer \textit{spawn} en cada una de les instàncies Erlang que haurem d'haver creat i en cada una d'elles fem que s'executi amb \textit{fun} la funció que vulguem. També necessitem canviar totes les referències a les instàncies, degut a que ara s'executaran en instàncies Erlang diferents i per tant no ens podem referir a elles com fins ara. Cada vegada que tinguem \textit{lx} haurem de fer-ho amb una tupla de l'estil de \textit{\{lx, 'nodex@127.0.0.1'\}}. 
\\
\textbf{iii) Resolving deadlock}\\
El tercer experiment realitzat consisteix en intentar eliminar les deadlock. Per gestionar aquesta situació donarem a cada instància de \textit{lock} un identificador i el passarem per paràmetre en cada funció. Amb això aconseguirem que puguem donar una prioritat a cada instància de \textit{lock}, llavors podem enviar un OK quan una instància de \textit{lock} ens faci un \textit{request} i tingui prioritat superior.\\
Per resoldre la situació conflictiva de quan un procés ja ha confirmat un altre procés amb prioritat menor que el que ho demana ara hem fet servir la solució proposada: enviar un missatge adicional de \textit{request} un cop enviat l'OK.

\end{itemize}

\begin{itemize}


\end{itemize}

\newpage

\section{Open questions}

\begin{itemize}



\textbf{Versió 1:}\\
\\
\textbf{i)What is the behavior of this lock when you increase
the risk of a conflict?\\}

\\
\textbf{Resolving Deadlock:}\\}
\\
\textbf{i) Justify how your code guarantees that only one
process is in the critical section at any time.\\}\\
\\
\textbf{ii)  What is the main drawback
of lock2 implementation?\\}\\ 
\\

\textbf{Lamport’s time:}\\}
\\
\textbf{i)  Would it be possible that a worker is given
access to a critical section prior to another worker that issued a request to its lock instance logically before (assuming happened-before order)?\\}\\
\\



\end{itemize}

\section{Personal opinion}

\item{}


\end{document}
